\documentclass{article}

\usepackage{graphicx}
\usepackage{hyperref}
\usepackage{tikz}
\usepackage{pgfplots}
\usepackage{multirow}
%\usepackage{indentfirst}
\pgfplotsset{compat=1.16}


\author{Roșu Cristian-Mihai}
\title{HW0 Report}

\begin{document}
\maketitle

\section{Introduction}
Implement a deterministic and a heuristic search method.
Test and compare them using 4 benchmark functions.
One of the functions should be \textbf{Rastrigin}'s Function:

$$ f(x) = A \cdot n + \sum_{i=1}^n \left[ x_i^2 - A \cdot cos(2 \pi x_i) \right],
A = 10, x_i \in \left[ -5.12, 5.15 \right]$$

The functions should be tested using the following numbers of dimensions:
2, 5, 20.
In the case of heuristic search methods, a proper sample size (minimally 30) should be used.

Write a report detailing and explaining your motivation, method, results, and draw conclusions.

\subsection{Motivation}
For the three other functions, apart from Rastrigin's, I chose \textbf{Michalewicz}'s, \textbf{Rosenbrock}'s and the \textbf{Sphere} function.

$$ f(x) = - \sum_{i=1}^n \left[ \sin(x_i^2) \cdot \sin(\frac{i{x_i}^2}{\pi})^{2m} \right],
m = 10, x_i \in \left[ 0, \pi \right]$$

$$ f(x) = \sum_{i=1}^{n-1} \left[ 100({x_i}^2-x_{i+1})^2+(x_i-1)^2 \right], x_i \in \left[ -5, 10 \right]$$

$$ f(x) = \sum_{i=1}^n {x_i}^2, x_i \in \left[ -5.12, 5.12 \right]$$

The reason why I chose those three specifically is because they are among the ones who can work with any number of variables, but more importantly because their codomain is of relatively small size, compared with the rest of the options.
Among them, Michalewicz's function was chosen especially for its unique global minima.
\section{Method}
The methods I used are as follows:

For the heuristic method, I decided to use a simple random search algorithm. It randomly generates a real number and compares the value obtained after passing it through the function to the variable holding the current minimum. This process is repeated through a million iterations in order to cover a large portion of the function's domain. In order to cut down on the time needed to perform these operations, this task is split among two threads, using two asynchronous functions, each calculating its own minimum over just 500,000 iterations. Two threads is an optimal choice, since more than that resulted in an improvement of just one second(i.e. 2 and 3 threads).

For the deterministic method, I decided on using a Brute-force search algorithm. It iterates through the entirety of the function's domain and similarly makes comparisons to find the minimum. Due to its nature, the algorithm works best in the 2-dimensional scenario, finding the solution in a huge amount of time in the other cases.

Because of this, I opted for a second, "lazier" approach, which also generates solutions by passing through the function's whole domain, but it does so by making all of the solution's components the same(i.e. for n=2 and x=[1,1], when n=3 then x=[1,1,1]). This brings satisfactory results only because of the four functions that are being tested, whose global minima are obtained from values in that form.

%\begin{figure}[!h]
%  \includegraphics[width=\textwidth,height=\textheight,keepaspectratio]{Rastrigin_function}
%  \caption{Rastrigin's Function. Axes should be labeled.}
%\end{figure}

\section{Experiment}
The experiment consists in simply running the algorithms once, in the case of the deterministic ones, and running the heuristic one 30 times in order to obtain consistent results which can be safely compared to our expectation. 

The results sought to obtain are the minima found by the algorithms an the times needed to reach those minima. Regarding the heuristic method, we are instead searching for the average of these we get from the 30 iterations.

The code for this can be found on Github.\cite{github}

\section{Results}
Below are 4 tables corresponding to each of the 4 functions that hold the results to the experiment: Rastrigin, Michalewicz, Rosenbrock and the Sphere function, in that order.

\newpage

\begin{figure}[!h]
%\begin{table}[ht]
\begin{tabular}{|c|l|l|l|l|}
\hline
Method & \multicolumn{1}{c|}{Dimension n} & \multicolumn{1}{c|}{2} & \multicolumn{1}{c|}{5} & \multicolumn{1}{c|}{20} \\ \hline
\multirow{2}{*}{Lazy} & Min & 0 & 0 & 0 \\ \cline{2-5} 
 & Time & 32ms & 56ms & 112ms \\ \hline
\multirow{2}{*}{Deterministic} & Min & 0 & - & - \\ \cline{2-5} 
 & Time & 1039ms & \textgreater{}30min & \textgreater{}\textgreater{}30min \\ \hline
\multirow{6}{*}{Nedeterministic} & Min & 0.000304872 & 3.54069 & 145.893 \\ \cline{2-5} 
 & Max & 0.0171115 & 9.37915 & 174.082 \\ \cline{2-5} 
 & Average & 0.00534376 & 6.93064 & 162.607 \\ \cline{2-5} 
 & Min time & 5030ms & 9226ms & 22925ms \\ \cline{2-5} 
 & Max time & 5319ms & 9486ms & 23629ms \\ \cline{2-5} 
 & Average time & 5128ms & 9304ms & 23087ms \\ \hline
\end{tabular}
%\end{table}
\caption{Rastrigin's function results}
\end{figure}

\begin{figure}[!h]
%\begin{table}[ht]
\begin{tabular}{|c|l|l|l|l|}
\hline
Method & \multicolumn{1}{c|}{Dimension n} & \multicolumn{1}{c|}{2} & \multicolumn{1}{c|}{5} & \multicolumn{1}{c|}{20} \\ \hline
\multirow{2}{*}{Lazy} & Min & -1.00096 & -2.38927 & -7.94934 \\ \cline{2-5} 
 & Time & 9ms & 17ms & 35ms \\ \hline
\multirow{2}{*}{Deterministic} & Min & -1.79779 & - & - \\ \cline{2-5} 
 & Time & 106ms & \textgreater{}30min & \textgreater{}\textgreater{}30min \\ \hline
\multirow{6}{*}{Nedeterministic} & Min & -1.80163 & -4.52292 & -9.53519 \\ \cline{2-5} 
 & Max & -1.80137 & -3.79666 & -7.96045 \\ \cline{2-5} 
 & Average & -1.80156 & -4.10483 & -8.58571 \\ \cline{2-5} 
 & Min time & 5263ms & 9477ms & 22805ms \\ \cline{2-5} 
 & Max time & 5402ms & 9814ms & 23314ms \\ \cline{2-5} 
 & Average time & 5307ms & 9554ms & 22984ms \\ \hline
\end{tabular}
%\end{table}
\caption{Michalewicz's function results}
\end{figure}

\begin{figure}[!h]
%\begin{table}[ht]
\begin{tabular}{|c|l|l|l|l|}
\hline
Method & \multicolumn{1}{c|}{Dimension n} & \multicolumn{1}{c|}{2} & \multicolumn{1}{c|}{5} & \multicolumn{1}{c|}{20} \\ \hline
\multirow{2}{*}{Lazy} & Min & $2.33877 \cdot 10^{-25}$ & $5.84692 \cdot 10^{-25}$ & $2.33877 \cdot 10^{-24}$ \\ \cline{2-5} 
 & Time & 31ms & 52ms & 99ms \\ \hline
\multirow{2}{*}{Deterministic} & Min & $8.56874 \cdot 10^{-28}$ & - & - \\ \cline{2-5} 
 & Time & 1033ms & \textgreater{}30min & \textgreater{}\textgreater{}30min \\ \hline
\multirow{6}{*}{Nedeterministic} & Min & $2.02137 \cdot 10^{-6}$ & 0.0329426 & 17.4257 \\ \cline{2-5} 
 & Max & $7.68284 \cdot 10^{-5}$ & 0.327834 & 45.3134 \\ \cline{2-5} 
 & Average & $2.43108 \cdot 10^{-5}$ & 0.179344 & 35.5202 \\ \cline{2-5} 
 & Min time & 4933ms & 8960ms & 21356ms \\ \cline{2-5} 
 & Max time & 5304ms & 9267ms & 21670ms \\ \cline{2-5} 
 & Average time & 5040ms & 9058ms & 21477ms \\ \hline
\end{tabular}
%\end{table}
\caption{Sphere function results}
\end{figure}

\begin{figure}[!h]
%\begin{table}[ht]
\begin{tabular}{|c|l|l|l|l|}
\hline
Method & \multicolumn{1}{c|}{Dimension n} & \multicolumn{1}{c|}{2} & \multicolumn{1}{c|}{5} & \multicolumn{1}{c|}{20} \\ \hline
\multirow{2}{*}{Lazy} & Min & 0 & 0 & 0 \\ \cline{2-5} 
 & Time & 42ms & 75ms & 155ms \\ \hline
\multirow{2}{*}{Deterministic} & Min & 0 & - & - \\ \cline{2-5} 
 & Time & 2186ms & \textgreater{}30min & \textgreater{}\textgreater{}30min \\ \hline
\multirow{6}{*}{Nedeterministic} & Min & $2.78091 \cdot 10^{-5}$ & 13.0901 & 33526 \\ \cline{2-5} 
 & Max & 0.0026726 & 55.7838 & 87850 \\ \cline{2-5} 
 & Average & 0.000686644 & 29.1294 & 62896 \\ \cline{2-5} 
 & Min time & 4723ms & 9202ms & 22150ms \\ \cline{2-5} 
 & Max time & 5125ms & 9711ms & 22462ms \\ \cline{2-5} 
 & Average time & 4825ms & 9307ms & 22299ms \\ \hline
\end{tabular}
%\end{table}
\caption{Rosenbrock's function results}
\end{figure}

\section{Conclusions}
The deterministic method is the most efficient out of the two, however that holds true only while we are working with a low number of dimensions. As we scale the dimension upwards, the Brute-force approach quickly proves to be way too inefficient time wise. On the other hand, although the heuristic method drifts away from the correct solution, its times are consistent across the different dimensions, and the different function in fact. 

\begin{thebibliography}{9}

\bibitem{github}
  Github repository for the project \\
  \url{https://github.com/Nenma/ga-hw0}
  
\bibitem{brute-force}
  Wikipedia page for Brute-force algorithms \\
  \url{https://en.wikipedia.org/wiki/Brute-force_search}
  
\bibitem{tutorial}
  Simple Latex tutorial I used \\
  \url{http://www.docs.is.ed.ac.uk/skills/documents/3722/3722-2014.pdf}

\end{thebibliography}  
\end{document}
